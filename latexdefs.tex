\usepackage[english]{babel}
\usepackage[utf8]{inputenc}
\usepackage[margin=1in]{geometry} 
\usepackage[square]{natbib}
\bibliographystyle{abbrvnat}

\usepackage[nottoc]{tocbibind}
\usepackage{amsmath, amsthm, amssymb, graphicx, tikz, hyperref, ifthen, xspace, bm, ulem}
\usepackage{fancyhdr}
%\addto\captionsenglish{\renewcommand{\chaptername}{Lecture}} % change Chapter to Lecture

% codes 
\usepackage{minted} % actual codes
\usepackage[ruled, algosection, vlined]{algorithm2e} % pseudo codes


% hyperref setup
\hypersetup{
    colorlinks=true,
    linkcolor=blue,
    filecolor=magenta,      
    urlcolor=cyan,
}

\usepackage[nameinlink,noabbrev]{cleveref}
\newcommand*{\fullref}[1]{\hyperref[{#1}]{\Cref*{#1} -- \nameref*{#1}}}

\parindent0pt

% itemization
\usepackage{enumerate}
\renewcommand{\labelitemii}{$\diamond$}
\renewcommand{\labelitemi}{$\triangleright$}
\renewcommand{\labelitemiii}{$\circ$}
\renewcommand{\labelitemiv}{$\ast$}


% definitions, theorems, lemmas, etc
\theoremstyle{plain}
\newtheorem{thm}{Theorem}[section]
\newtheorem{cor}[thm]{Corollary}
\newtheorem{lemma}[thm]{Lemma}
\newtheorem{prop}[thm]{Proposition}

\theoremstyle{definition}
\newtheorem{defn}[thm]{Definition}
\newtheorem{example}[thm]{Example}
\newtheorem{ass}{Assumption}

\theoremstyle{remark}
\newtheorem*{claim}{Claim} 
\newtheorem{rem}{Remark}[section]

% Multiline comments
\newcommand{\comment}[1]{}

% Number sets
\newcommand{\N}{\mathbb{N}}
\newcommand{\Z}{\mathbb{Z}}
\newcommand{\Q}{\mathbb{Q}}
\newcommand{\R}{\mathbb{R}}	
\newcommand{\C}{\mathbb{C}}

% Mathcal
\newcommand{\cx}{\mathcal{X}}
\newcommand{\cf}{\mathcal{F}}

% Analysis
\newcommand{\set}[1]{\left\lbrace #1 \right\rbrace}
\newcommand{\where}{\ \middle\vert\;}
\newcommand{\abs}[1]{\left\vert #1 \right\vert}
\newcommand{\norm}[2][]{{\left\Vert #2 \right\Vert}_{#1}}

%\let + \newname + \old name to rename command
%\renewcommand{•}{•} overrides predefined commands
\let\dbar\d \renewcommand{\d}{\mathrm{d}}
\newcommand{\D}{\,\mathrm{d}}
\newcommand{\deriv}[2]{\frac{\d #1}{\d #2}}
\newcommand{\pd}[2]{\frac{\partial #1}{\partial #2}}
\DeclareMathOperator*{\argmaxRM}{arg\,max}
\DeclareMathOperator*{\argminRM}{arg\,min}
\newcommand{\seq}[2][n]{(#2_{#1})_{#1 \geq 0}}
\newcommand{\e}{\mathrm{e}}

% Piecewise defined function
\newcommand{\funcdef}[1]{\left\lbrace \hspace{-1.2mm}
	\setlength{\arraycolsep}{3mm} \begin{array}{ll} #1 \end{array}
	\setlength{\arraycolsep}{0mm} \hspace{-1.8mm} \right.}

% Linear Algebra
\newcommand{\vect}[1]{\bm{#1}}
\newcommand{\matr}[1]{\bm{#1}} 
\newcommand{\tr}[1]{#1^{\top}}
\renewcommand{\sp}[2]{#1^{\top}#2}   %scalar product
\newcommand{\xmatrix}[1]{\!\left(\:\begin{matrix} #1 \end{matrix}\:\right)\!}	
\newcommand{\mpsi}[1]{(#1^{\top}#1)^{-1}#1^{\top}}
\newcommand{\grad}[1]{\mathrm{grad}(#1)}
\newcommand{\curl}[1]{\mathrm{curl}(#1)}
\newcommand{\Div}[1]{\mathrm{div}(#1)}
\newcommand{\sign}[1]{\mathrm{sign}(#1)}
\newcommand{\rank}[1]{\mathrm{rank}(#1)}

% Indicator function
\newcommand{\one}{1}
\newcommand{\indicator}[1]{\one_{\left\lbrace #1\right\rbrace}}
\DeclareRobustCommand\1{\futurelet\oneNext\oneCheck}%
\def\oneCheck{%
	\ifx\bgroup\oneNext \expandafter\indicator%
	\else%
	\expandafter\one%
	\fi%
}


% Probability
\renewcommand{\P}[1]{\mathbb{P}\left(#1\right)}
\newcommand{\E}[1]{\mathbb{E}#1}
\newcommand{\EXP}[2][]{\mathbb{E}_{{#1}}\left[#2\right]}
\newcommand{\given}{\;\!\middle\vert\,}
\newcommand{\F}{\mathcal{F}}
\newcommand{\filt}{\mathbb{F}}
\newcommand{\cpm}[1]{{#1}^c}
\newcommand{\cov}[1]{\text{cov}{(#1)}}
\newcommand{\kur}[1]{\text{kur}{(#1)}}
\newcommand{\var}[1]{\text{Var}{\left(#1\right)}}
\newcommand{\corr}[1]{\rho{(#1)}}
\newcommand{\freq}[1]{\text{freq}\left(#1\right)}

% Probability Distributions
\newcommand{\dber}[1]{\mathrm{Bern}(#1)} %Bernoulli distribution
\newcommand{\dbeta}[1]{\mathrm{Beta}(#1)} %Beta distribution
\newcommand{\dbin}[1]{\mathrm{Bin}(#1)} %Binomial distribution 
\newcommand{\ddir}[1]{\mathrm{Dir}(#1)} %Dirichlet distribution
\newcommand{\dgamma}[1]{\mathrm{Gam}(#1)} %Gamma distribution
\newcommand{\dnorm}[1]{\mathcal{N}(#1)} %Normal/Gaussian distribution
\newcommand{\dst}[1]{\mathrm{St}(#1)} %Student t distribution
\newcommand{\duni}[1]{\mathrm{U}(#1)} %Uniform distribution
\newcommand{\dWis}[1]{\mathcal{W}(#1)} %Wishart distribution


% Abbreviations
\newcommand{\wrt}{with respect to\xspace}
\newcommand{\ie}{i.e.}
\newcommand{\Ps}{\mathbb{P}}
\newcommand{\iid}{i.i.d }
\newcommand{\soln}{\textit{Solution}.}
